\documentclass[11pt,a4paper,fleqn]{scrartcl}

\usepackage[utf8]{inputenc}
\usepackage[T1]{fontenc}
\usepackage[colorlinks=true, citecolor=blue, linkcolor=blue, filecolor=blue,urlcolor=blue]{hyperref}
\hypersetup{
     colorlinks   = true,
     citecolor    = gray
}
\usepackage{wrapfig}

\usepackage{caption}
\captionsetup{format=plain, indent=5pt, font=footnotesize, labelfont=bf}

\setkomafont{disposition}{\scshape\bfseries}

\usepackage{amsmath}
\usepackage{amssymb}
\usepackage{amsfonts}
\usepackage{bbm}
\usepackage{mathtools}
% \usepackage{epsfig}
% \usepackage{grffile}
%\usepackage{times}
%\usepackage{babel}
\usepackage{tikz}
\usepackage{paralist}
\usepackage{color}
\usepackage[top=3cm, bottom=2.5cm, left=2.5cm, right=3cm]{geometry}
%\setlength{\mathindent}{1ex}

% PGF
\usepackage{pgfplots}
\usepackage{pgf}
\usepackage{siunitx}
\usepackage{xfrac}
\usepackage{calculator}
\usepackage{calculus}
\usepackage{eurosym}
\usepackage{booktabs}
%\sisetup{per-mode=fraction,%
%	fraction-function=\sfrac}

%\newcommand{\eur}[1]{\EUR{#1}\si{\per\kilo\meter}}
\pgfplotsset{
  compat=newest,
  every axis/.append style={small, minor tick num=3}
}

%\usepackage[backend=biber,style=alphabetic,url=false,doi=false]{biblatex}
%\addbibresource{sheet01_biber.bib}
% \addbibresource{/home/coroa/papers/refs.bib}

\newcommand{\id}{\mathbbm{1}}
\newcommand{\NN}{{\mathbbm{N}}}
\newcommand{\ZZ}{{\mathbbm{Z}}}
\newcommand{\RR}{{\mathbbm{R}}}
\newcommand{\CC}{{\mathbbm{C}}}
\renewcommand{\vec}[1]{{\boldsymbol{#1}}}

\renewcommand{\i}{\mathrm{i}}

\newcommand{\expect}[1]{\langle\,#1\,\rangle}
\newcommand{\e}[1]{\ensuremath{\,\mathrm{#1}}}

\renewcommand{\O}{\mc{O}}
\newcommand{\veps}{\varepsilon}
\newcommand{\ud}[1]{\textup{d}#1\,}

\newcommand{\unclear}[1]{\color{green}#1}
\newcommand{\problem}[1]{\color{red}#1}
\newcommand{\rd}[1]{\num[round-mode=places,round-precision=1]{#1}}

%\DeclareSIUnit{\euro}{\EUR}
\DeclareSIUnit{\dollar}{\$}
\newcommand{\eur}{\text{\EUR{}}}

\newcommand{\lgans}{\textquotedblleft}
\newcommand{\rgans}{\textquotedblright}
\newcommand{\D}{\Delta}

\def\l{\lambda}
\def\m{\mu}
\def\d{\partial}
\def\nn{\nonumber}


\newcommand{\ubar}[1]{\text{\b{$#1$}}}


\newcommand{\E}{\mathbb{E}}
\newcommand{\dx}[1]{\mbox{d}{#1}}
\renewcommand{\vec}[1]{{\bf #1}}

\usepackage{palatino}
\usepackage{mathpazo}
\setlength\parindent{0pt}
\usepackage{xcolor}
\usepackage{framed}
\definecolor{shadecolor}{rgb}{.9,.9,.9}

%=====================================================================
%=====================================================================
\begin{document}

\begin{flushright}
  \textbf{Energy System Modelling }\\
  {\small Karlsruhe Institute of Technology}\\
  {\small Institute for Automation and Applied Informatics}\\
  {\small Summer Term 2020}\\
 \end{flushright}
 
  
  \vspace{-0.5em}
  \hrulefill
  \vspace{0.3em}
 
 \begin{center}
  \textbf{\textsc{\Large Solution V: Investment and Large Power Systems}}\\
  \small Will be worked on in the exercise session on Thursday, 26 June 2020.\\[1.5em]
 \end{center}
 
 \vspace{-0.5em}
 \hrulefill
 \vspace{0.8em}

%=============== ======================================================
\paragraph{Solution V.1 \normalsize (investment, generator and transmission constraints).}~\\
%=====================================================================

Two generators are connected to the grid by a single transmission
line (with no electrical demand on their side of the transmission line). A single company owns both the generators and the transmission line. Generator 1 has a linear cost curve $C_1(g_1) = 5 g_1$ [\euro/h] and a capacity of 300~MW and Generator 2 has a linear cost curve $C_2(g_2) = 10 g_2$ [\euro/h] and a capacity of 900~MW. The transmission line has a capacity $K$ of 1000~MW. Suppose the demand in the grid is always high enough to absorb the
generation from the two generators and that the market price of
electricity $\pi$ is never below 15 \euro/MWh and averages 20
\euro/MWh.

\begin{enumerate}[(a)]
 \begin{shaded}\item Determine the full set of equations (objective function and
  constraints) for the generators to optimise their dispatch to
  maximise total economic welfare.\end{shaded}

 Note that it is important in this example that the same company owns
 both the generators and the transmission line; if an independent TSO
 owned the transmission line, he could take the congestion revenue for
 himself.

 If we label the dispatch of Generator 1 by $g_1$ and of Generator 2 by $g_2$, then the objective function is to maximise total profit
 \begin{equation*}
  \max_{g_1,g_2} \left[ \pi (g_1+g_2) - C_1(g_1) - C_2(g_2) \right] =    \max_{g_1,g_2} \left[ \pi (g_1+g_2) - 5g_1 - 10g_2 \right]
 \end{equation*}

 The constraints are
 \begin{align*}
  g_1     & \leq \hat g_1 & \leftrightarrow & \bar{\m}_1  \\
  -g_1    & \leq 0        & \leftrightarrow & \ubar{\m}_1 \\
  g_2     & \leq \hat g_2 & \leftrightarrow & \bar{\m}_2  \\
  -g_2    & \leq 0        & \leftrightarrow & \ubar{\m}_2 \\
  g_1+g_2 & \leq K        & \leftrightarrow & \m_T
 \end{align*}
 Where the first four constraints come from generation, where $\hat g_1 = $ 300 MW and $\hat g_1 = $ 900 MW and the final constraint comes from the transmission, where $K = $ 1000~MW is the capacity of the export transmission line.

 \begin{shaded}\item What is the optimal dispatch?\end{shaded}

 Since the market price is always higher than the marginal price
 of the generators, they will both run as high as possible given the
 constraints. Since Generator 1 is cheaper than Generator 2, it will
 max-out its capacity first, so that $g_1^* = \hat g_1 =$ 300~MW. Generator 2 will output as much as it can given the transmission constraint, so that $g_2^* =$ 700~MW.

 \begin{shaded}\item What are the values of the KKT multipliers for all the constraints in terms of $\pi$?\end{shaded}

 From stationarity we have for $g_1$ the non-zero terms:
 \begin{align*}
  0 & =   \frac{\d}{\d g_1} \left( \pi (g_1+g_2) - 5g_1 - 10g_2\right) - \bar{\m}_1 \frac{\d}{\d g_1} (g_1-\hat g_1)- \ubar{\m}_1 \frac{\d}{\d g_1} (-g_1) -\m_T \frac{\d}{\d g_1} (g_1+g_2-K)  \nn \\
    & = \pi -5 - \bar{\m}_1 + \ubar{\m}_1 - \m_T
 \end{align*}
 For $g_2$ we have
 \begin{align*}
  0 & =   \frac{\d}{\d g_2} \left( \pi (g_1+g_2) - 5g_1 - 10g_2\right) - \bar{\m}_2 \frac{\d}{\d g_2} (g_2-\hat g_2)- \ubar{\m}_2 \frac{\d}{\d g_2} (-g_2) -\m_T \frac{\d}{\d g_2} (g_1+g_2-K)  \nn \\
    & = \pi - 10- \bar{\m}_2 + \ubar{\m}_2 - \m_T
 \end{align*}

 At the optimal point we can see that $\ubar{\m}_1$, $\bar{\m}_2$ and $\ubar{\m}_2$ are non-binding, so these are zero. To solve for $\m_T$ and $\bar{\m}_1$ we have two equations:
 \begin{align*}
  0 & = \pi - 5 - \bar{\m}_1 - \m_T \nn \\
  0 & = \pi - 10  - \m_T
 \end{align*}
 Therefore
 \begin{align*}
  \m_T       & = \pi - 10 \\
  \bar{\m}_1 & = 5
 \end{align*}

 \begin{shaded}\item A new turbo-boosting technology can increase the capacity of Generator 1 from 300~MW to 350~MW.  At what annualised capital cost would it be efficient to invest in this new technology?\end{shaded}

 The value of $\bar{\m}_1$ gives us the increase in profit for a small increase in $\hat{g}_1$. We want to understand a large increase in $\hat{g}_1$ of 50 MW, therefore we have to integrate over $\bar{\m}_1$ as a function of $\hat{g}_1$, since the value of $\bar{\m}_1$ may change as $\hat{g}_1$ changes. The total increase in profitability for expanding $\hat{g}_1$ from 300~MW to 350~MW is then
 \begin{equation*}
  \int_{300}^{350} \bar{\m}_1(\hat{g}_1) d\hat{g}_1
 \end{equation*}
 Because of the linearity of the problem,  $\bar{\m}_1$ is actually constant as we expand $\hat{g}_1$ in the region from 300~MW to 350~MW. The extra profit would be per year: 5 \euro/MWh * 50 MW * 8760h/a = \euro 2.19 million/a.
 At or below this annualised capital cost, it would be worth investing.

 \begin{shaded}\item A new high temperature conductor technology can increase the capacity of the transmission line by 200~MW. At what annualised capital cost would it be efficient to invest in this new technology?\end{shaded}

 Here $\m_T$ changes as $K$ is expanded, so we have to integrate:
 \begin{equation*}
  \int_{1000}^{1200} \m_T(K) dK
 \end{equation*}
 Since  $\m_T$ is constant as we expand $K$ from 1000~MW to 1200~MW, the extra profit would be per year: (average$(\pi)$-10) \euro/MWh * 200 MW * 8760h/a = \euro 17.52 million/a.
 At or below this annualised capital cost, it would be worth investing. An extension beyond 1200~MW would not bring any benefit, because the generator constraints would be then binding.
\end{enumerate}
%=============== ======================================================
\paragraph{Solution V.2 \normalsize (duration curves and generation investment).}~\\
%=====================================================================

Let us suppose that demand is inelastic. The demand-duration curve is given by $D=1000-1000z$, where $z\in[0,1]$ represents the probability of time the load spends above a certain value. Suppose that there is a choice between coal and gas generation plants with a variable cost of 2 and 12~\euro/MWh, together with load-shedding at 1012\euro/MWh. The fixed costs of coal and gas generation are 15 and 10~\euro/MWh, respectively.

\begin{enumerate}[(a)]
 \begin{shaded}
  \item Describe the concept of a screening curve and how it helps to determine generation investment, given a demand-duration curve.
 \end{shaded}

 A screening curve plots the costs of different generators as a
 function of their utilization/capacity/usage factor so that they can
 be compared based on their fixed and variable costs. The utilization
 factor is plotted along the $x$ axis from 0 to 1, 0 corresponding to
 no running time, 1 corresponding to the power plant running 100\% of
 the time. The intercept of the curve of each generator with the $y$
 axis is given by the fixed cost $f$ [\euro/MWh] (i.e. the cost with no
 variable costs) and the slope is given by the variable cost $c$
 [\euro/MWh].

 The interception points of the linear curves of the different
 generators determine the ranges of utilization factors in which one
 generator is cheaper than another. By comparing the screening curves with the demand duration curve, the
 correct generator capacities for different utilisation factors can be
 determined (e.g. how much baseload power is required, how much peaking
 power is required, how much load shedding).

 \begin{shaded}\item Plot the screening curve and find the intersections of the generation technologies.\end{shaded}
 First we work out the intersection points of
 the generators as a function of their capacity factors, then we work out the
 capacities $K_*$ of the generators.

 The screening curves tell us above which capacity factor it costs less
 to run one type of generator rather than another.


 \begin{table}[!h]
  \centering
  \begin{tabular}{lrr}
   \toprule
   Generator     & $c_i$ [\euro/MWh] & $f_i$ [\euro/MWh] \\
   \midrule
   coal          & 2                 & 15                \\
   gas           & 12                & 10                \\
   load-shedding & 1012              & 0                 \\
   \bottomrule
  \end{tabular}
 \end{table}

 Generators coal and gas intersect at $x_{cg}$ given by
 \begin{equation*}
  15 + 2x_{cg} = 10 + 12x_{cg}
 \end{equation*}
 i.e. $x_{cg}=0.5$. This means that if the coal generator can run more than 50\% of the time, it should be built from an economic perspective.

 Gas generator and load-shedding intersect at $x_{gl}$ given by
 \begin{equation*}
  10 + 12x_{gl} = 1012x_{gl}
 \end{equation*}
 i.e.\ $x_{gl}=1/100$. This means that for 1\% of the time we have load-shedding because it's not economical to cover the rare times of very high load.

 \begin{shaded}\item Compute the long-term equilibrium power plant investment (optimal mix of generation) using PyPSA.\end{shaded}

 The amount of load that is present at least $x_{cg}$ of the time determines $K_{coal}$, which we find by solving based on the load duration curve
 \begin{equation*}
  1000-1000x_{cg} = K_{coal} \xRightarrow{x_{cg}=0.5} K_{coal} = 500
 \end{equation*}

 To get the capacity of the gas generator we solve based on the load duration curve
 \begin{equation*}
  1000-1000x_{gl}=K_{coal}+ K_{gas} \xRightarrow{x_{cg}=0.5 \text{ and } K_{coal}=500} K_{gas} = 490
 \end{equation*}

 Load above $K_{coal}+ K_{gas}=990$ is shed. Thus, $K_{load-shedding}=10$.

 \begin{shaded}\item Plot the resulting price duration curve and the generation dispatch. Comment!\end{shaded}
 cf.\ Jupyter Notebook
 \begin{shaded}\item Demonstrate that the zero-profit condition is fulfilled.\end{shaded}
 cf.\ Jupyter Notebook
 \begin{shaded}
  \item While it can be shown that generators recover their cost in theory, name reasons why this might not be the case in reality.
 \end{shaded}
 Several factors make this theoretical picture quite different in reality:
 \begin{itemize}
  \item Generation investment is lumpy; i.e. you can often only
        build power stations in e.g. 500~MW blocks, not in continuous chunks.
  \item Some older generators have sunk costs, i.e. costs which have been incurred once and cannot be recovered, which alters their behaviour (i.e. the $f$ term is not evenly distributed across all hours)
  \item Returns on scale in building plant are not taken into account (we did everything linear)
  \item Site-specific concerns ignored (e.g. lignite might need to be near a mine and have limited capacity)
  \item Variability of production for wind/solar ignored
  \item There is considerable  uncertainty given load/weather conditions during a year, which makes investment risky; economic downturns reduce electricity demand
  \item Fuel cost fluctuations, building delays which cost money
  \item Risks from third-parties:  Changing costs of other generators, political risks (carbon taxes,  Atomausstieg, subsidies for renewables, price caps)
  \item Political or administrative constraints on wholesale energy
        prices may prevent prices from rising high enough for long enough
        to justify generation investment (``Missing Money Problem'')
  \item  Lead-in time for planning and building, behaviour of others, boom-and-bust investment cycles resulting from periods of under- and over-investment in capacity
  \item Exercise of market power
 \end{itemize}
\end{enumerate}







%=============== ======================================================
\paragraph{Solution V.3 \normalsize (synthetic fuels).}~\\
%=====================================================================

cf. Jupyter Notebook

%=============== ======================================================
%\paragraph{Solution V.4 \normalsize (network clustering).}~\\
%=====================================================================

\end{document}
